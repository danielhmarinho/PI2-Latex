\begin{anexosenv}

\partanexos

\chapter{Plano de Gerenciamento de Riscos}

\section{Objetivos do Gerenciamento de Riscos}

Este documento tem como objetivo fazer uma análise dos riscos existentes dentro do projeto do biorreator automatizado para fermentação alcoólica e posteriormente buscar alternativas para mitigar ou eliminar os riscos identificados.

\section{Planejar Gerenciamento de Riscos}

Este processo define as atividades que servirão de auxílio para gerenciar os riscos durante todo o projeto:

\begin{itemize}
\item Identificar riscos do projeto através de reuniões, ferramentas e técnicas como \textit{brainstorm},  posteriormente documentando os mesmos.
\item Listar todos os riscos identificados para os integrantes do projeto.
\item Planejar soluções ou respostas para que os riscos identificados não impactem de forma abrupta o projeto.
\item Realizar o gerenciamento dos riscos identificados na atividade anterior.
\item Monitorar e controlar os riscos durante as reuniões presenciais do grupo para que os mesmos possam ser mitigados ou eliminados.
\end{itemize}

\section{Identificar Riscos}

Com o intuito de facilitar a identificação dos riscos e o compartilhamento dos mesmos com toda a equipe, foi utilizada a técnica conhecida como brainstorming. Nesta técnica os membros discutem qualquer tipo de idéia a cerca de um determinado tema. A tabela abaixo representa os riscos identificados no projeto durante as reuniões:




\end{anexosenv}
