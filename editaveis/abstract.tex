\begin{resumo}[Abstract]
 \begin{otherlanguage*}{english}
   One of the biggest environmental problems nowadays is the use of fossil fuels and so it’s necessary to unlink this kind of fuel from world energy matrix. The use of biofuels is growing and so its demand, thus the technologies associate with this subject are improving and becoming a very important world issue. In this scenario, ethanol has become an important biofuel due its Brazilian pioneering and because there is a consolidation in its industry which consequently is associated to the international economy. Thus the studies of this  process and its optimization have great importance to academic researches. The fermentation is considered the main step at obtaining this biofuel. It occurs when sugars are converted to alcohol and this process is made by microorganisms with a great sensitivity. Obtaining the best efficiency in this process depends on controlling parameters like temperature, pH and density. So, this paper work will carry out a construction project of a bench bioreactor totally automated with PID temperature controller and pH and density monitoring which will be controlled remotely by an app during the entire reaction process.


   \vspace{\onelineskip}

   \noindent
   \textbf{Key-words}: bioreactor. fermentation. automation. control. PID.

 \end{otherlanguage*}
\end{resumo}
