\begin{resumo}
  A necessidade da desvinculação do uso de combustíveis fósseis da matriz energética mundial, aliada à crescente demanda de biocombustíveis, faz com que o aprimoramento das tecnologias de obtenção desses combustíveis ganhem importância. Nesse cenário, o etanol tem destaque devido ao pioneirismo brasileiro de sua produção e consolidação no mercado internacional, de forma que o estudo e a otimização de seus processos de produção tem importância, principalmente para o desenvolvimento de pesquisas acadêmicas. A fermentação é considerada a principal etapa na obtenção desse biocombustível. Ela ocorre a partir da conversão de açúcares em álcool, realizada por microrganismos de grande sensibilidade, de modo que para que ocorra uma boa fermentação é necessário que os parâmetros do meio, como temperatura e pH estejam ajustados. Tendo isso em vista, neste trabalho será realizado o projeto e a construção de um biorreator de bancada automatizado com controle PID de temperatura, monitoramento de pH e densidade, que permitirá o controle de forma remota, via aplicativo, desses parâmetros enquanto a reação ocorre.


 \vspace{\onelineskip}

 \noindent
 \textbf{Palavras-chaves}: biorreator. fermentação. automatização. controle. PID.
\end{resumo}
